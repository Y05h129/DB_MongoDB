% !TeX root = ..\main.tex

\section{Installation}
There is a significant list of packages that you can download on the website of MongoDB: \url{www.mongodb.com}. 
Depending on operating system and processor of your server, you can choose a version and release channel of MongoDB. \parencite{Membrey2014}.

\subsection{Requirements e.g. configfiles}
The most important part is the operating system, just as MacOS, Solaris, Windows, Linux, etc. 
The second part, it depends on, is the bit version of your  system: 32-bit or 64-bit.
Important: Not every system supports every bit-version.
But the most important thing is the software itself, just as I wrote at the beginning.
The different versions:
\begin{itemize}
    \item Community Server (open-source version of MongoDB)
    \item Enterprise Server (the commercial version of MongoDB, which is available with a paid subscription)
\end{itemize}
and release channels:
\begin{itemize}
    \item Stable Release ()
    \item Release Candidate ()
    \item Development Release ()
\end{itemize}

For guarantying that every function and feature can be fully used, there are some supported platforms, with a list including all information, so you know if a version is available on your system.
You can view these information on the MongoDB installation page in the docs: \url{https://www.mongodb.com/docs/manual/installation/}.\parencite{Membrey2014}.

MongoDB supports a few platforms: 
\begin{itemize}
    \item Amazon Linux V1 and V2
    \item Debian 9, 10, 11
    \item RHEL/CentOS, Oracle Linux, Rocky, Alm a 9.0+, 8.0+, 7.0+, 6.2+
    \item SLES 12, 15
    \item Ubuntu 22.04, 20.04, 18.04, 16.04
    \item Windows Server 2019, 2008/2012/2012R2, 2016
    \item Windows 10
    \item macOS 11, 10.14+, 10.13, 10.12, 
\end{itemize}\parencite{Mongo-Installation}

\subsection{Linux server}

\subsubsection{Installation through Repositories}
So if you want to download MongoDB on Linux, you have got two different ways to go:
The first one installs MongoDB through the Repositories, it will install the packages automatically. 
In Ubuntu the default repositories includes MongoDB, the version might be out-of-date, so here is the installation-guide with \texttt{apt-get}:
You have to run a few lines in your repository-list in \texttt{/etc/apt/sources.list}.
\\
The first one to go is: \\
\texttt{deb https://downloads-distro.mongodb.org/repo/ubuntu-upstart dist 10gen}        % Zeilenumbruch checken, weiß aber auch nimmer wie das geht
Then you need to import MongoDB Inc's public GPG-key, which is used to sign the packages you want to download and ensure their consistency. 
To do that you can use the following apt-key command: 
\texttt{\$ apt-key adv --keyserver keyserver.ubuntu.com --recv 7F0CEB10}
To get the new repositories, you need to update it:
\texttt{\$ sudo apt-get update}
To finally install the software itself, you need to use this line in your shell:
\texttt{\$ sudo apt-get install mongodb-org}
To install a specific version, you can just add the version number, just like this:
\texttt{\$ sudo apt-get install mongodb-org=2.7.8}
And that's it, now you can use MongoDB, just as you want\parencite{Membrey2014}.

\subsubsection{Installation Manually}



\subsection{Docker}

\subsection{optional: Windows/MacOS}


\parencite{Mongo-Installation}

\parencite[p. ]{Subramanian2019}