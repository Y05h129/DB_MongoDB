% !TeX root = ..\main.tex

\section{Installation}
There is a significant list of packages that you can download on the website of MongoDB: \url{www.mongodb.com}. MongoDB provides a Community Edition as well as an Enterprise Edition \parencite{Mongo-Installation}.

\subsection{Installation requirements}
The MongoDB server's options and settings are specified using configuration files, which are YAML-files containing the relevant parameters. 
These configuration files allow for easy management and specification of options for the database server \parencite{Mongo-configuration}.

For guarantying that every function and feature can be fully used, there are some supported platforms, with a list including all information, so you know if a version is available on your system.
You can view this information on the MongoDB installation page in the docs: \url{https://www.mongodb.com/docs/manual/installation/}.
MongoDB can be installed natively on the following platforms: 
\begin{itemize}
    \item Amazon Linux V1 and V2
    \item Debian 9, 10, 11
    \item RHEL/CentOS, Oracle Linux, Rocky, Alm a 9.0+, 8.0+, 7.0+, 6.2+
    \item SLES 12, 15
    \item Ubuntu 22.04, 20.04, 18.04, 16.04
    \item Windows Server 2019, 2008/2012/2012R2, 2016
    \item Windows 10
    \item macOS 11, 10.14+, 10.13, 10.12
\end{itemize} \parencite{Mongo-Installation}

\subsection{Installation on Linux and Docker}
The preferred way to install MongoDB natively on a Linux desktop or server is by using the packages
provided by MongoDB. MongoDB also provides guides detailing the installation process \parencite{Mongo-linux-Install}.

Both editions of MongoDB can also be run in a Docker Container. To that end, MongoDB maintains the respective Docker images
and provides a detailed installation guides \parencite{Install-Mongo-Docker, Install-Mongo-Docker2}.
