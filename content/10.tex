% !TeX root = ..\main.tex

\section{Introduction to Document DBs}


\subsection{General information and characteristics}
Document-oriented databases are a one type of NoSQL databases that have gained popularity in recent years \parencite[p. 487]{Bach2016}. 
Unlike relational databases, where data is organized in tables, document-oriented databases store data as documents in collections \parencite[p. 493]{Bach2016}.

In general these documents and collections are \enquote{schemaless}, which means that the structure of documents can be flexible \parencite[p. 493]{Bach2016}.
This property facilitates agility in application development, particularly in applications with a constantly changing data model \parencite[p. 486]{Bach2016}.
Another strength of document-oriented databases is their ability to manage large amounts of data \parencite[p. 487]{Bach2016}. 
Document-oriented databases can distribute large amounts of data across multiple servers to ensure high scalability. 
This makes it possible to process large amounts of heterogeneous data \parencite[p. 229]{Meier2016}.

\subsection{History}
A basic idea behind the development of document based databases is to store all data of an object in the same place, i.e. in a single document \parencite{Faeskorn-Woyke2013}. 
One of the first document oriented databases is Lotus Notes from IBM, which has been developed in 1984 \parencite{Faeskorn-Woyke2013}.

The development of modern document-oriented databases began in the late 2000s as part of the NoSQL 
movement, which focused on processing large amounts of data \parencite[p. 41]{Eisermann2022}. 
MongoDB, CouchDB, and Firebase are among the significant document-oriented databases and 
are among the top 14 databases according to the 2021 StackOverflow survey \parencite[p. 41]{Eisermann2022}. 
