% !TeX root = ..\main.tex

\section{Introduction to Document DBs}


\subsection{General information and characteristics}
Document-oriented databases are one of the many categories of NoSQL databases used for managing 
data in modern applications. 
The focus of document-oriented databases is on storing data in the form of documents, 
which consist of a collection of key-value pairs. The data types that can be stored in a document 
are diverse and can include text, numbers, arrays, and other documents. 
Unlike relational databases, document-oriented databases do not have a fixed schema, 
which means they are more flexible and adaptable.

Another important characteristic of document-oriented databases is their horizontal scalability. 
This means they can be distributed across multiple servers to provide better performance and 
availability. By using sharding techniques, document-oriented databases can distribute 
data horizontally across multiple servers to achieve higher capacity and better scalability.

Document-oriented databases also support a variety of query methods, including indexing and 
aggregation. Indexing is a process where specific data fields in a document are indexed to 
make queries faster. Aggregation refers to grouping documents to calculate results, such as 
the number of documents that match a particular criterion.

Another important aspect of document-oriented databases is scaling read and write access. 
Some document-oriented databases support scaling write access by asynchronously replicating 
data to multiple servers. This means write access can be performed on one server instance while 
data is automatically replicated to other servers. Read access can also be scaled by 
distributing read queries to multiple servers to provide better performance and availability.

Some of the most well-known document-oriented databases include MongoDB, CouchDB, and RavenDB. 
MongoDB is one of the most popular document-oriented databases, known for its flexibility, 
high performance, and horizontal scalability.


\subsection{History}

Document-oriented databases originated in the 1980s with the introduction of Lotus Notes, 
which is considered a pioneer in this area. Lotus Notes was one of the first databases to use 
documents as its central data model. In the 1990s, other document-oriented databases emerged, 
such as ObjectStore and InterSystems Caché.

However, the modern era of document-oriented databases began in the 2000s, as the need for 
scalable databases to manage big data in web applications and other modern applications grew. 
The first document-oriented database developed in this context was CouchDB, which was created by 
Damien Katz in 2005. CouchDB was known for its ability to provide high availability and 
reliability while maintaining flexibility and scalability.

However, the modern era of document-oriented databases began in the 2000s, as the need for 
scalable databases to manage big data in web applications and other modern applications grew. 
The first document-oriented database developed in this context was CouchDB, which was created by 
Damien Katz in 2005. CouchDB was known for its ability to provide high availability and 
reliability while maintaining flexibility and scalability.

Since the introduction of CouchDB and MongoDB, document-oriented databases have become an 
important part of the modern database landscape. They provide developers with a powerful and 
flexible way to store and manage data, particularly in applications with a lot of unstructured 
data. Additionally, they enable companies to build scalable and reliable database infrastructures 
that meet the requirements of modern applications.

