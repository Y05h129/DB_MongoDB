% !TeX root = ..\main.tex

\section{Introduction to Document DBs}


\subsection{General information and characteristics}
Document-oriented databases are a special type of NoSQL databases that have gained popularity in recent years \parencite[p. 487]{Bach2016}. 
Unlike relational databases, where data is organized in tables, document-oriented databases  store data as unstructured documents \parencite[p. 493]{Bach2016}.
One of the main advantages of document-oriented databases is their flexibility in storing data \parencite[p. 488]{Bach2016}. Unlike relational databases, 
where the schema must be defined in advance, data can be stored in document-oriented databases without a fixed schema. 
This means that it is possible to add new data fields and structures without changing the database design. 
This property facilitates agility in application development, particularly in applications with a constantly changing data model \parencite[p. 486]{Bach2016}.
Another strength of document-oriented databases is their ability to manage large amounts of data. 
Document-oriented databases can distribute large amounts of data across multiple servers to ensure high scalability. 
This makes it possible to process large amounts of data in a short time while providing high performance \parencite[p. 487]{Bach2016}.
One of the most well-known document-oriented databases is MongoDB. MongoDB is an open-source database that is used in many application areas. 
MongoDB offers a variety of features, including the ability to perform complex aggregations \parencite[p. 495]{Bach2016}.
The use of document-oriented databases, which fall under the category of NoSQL databases, offers many advantages 
in agile development and adaptation to dynamic environments. 
However, they also present some challenges in application development and NoSQL data design. 
The lack of standardization in query languages and the flexibility of the database schema make it difficult to develop database-independent applications. 
When structuring NoSQL data, both planned queries and evaluations and the peculiarities of the used NoSQL DBMS must be considered. 
A systematic schema management can simplify the process of evolution of NoSQL data. 
Developers therefore require comprehensive knowledge of database implementation techniques to be able to use NoSQL DBMS correctly and in the long term \parencite[p. 441]{Klettke2016}.
Overall, document-oriented databases offer many advantages over relational databases, particularly in terms of flexibility and scalability.



\subsection{History}

The development of document-oriented databases began in the late 2000s as part of the NoSQL 
movement, which focused on processing large amounts of data. There are four main types of NoSQL 
databases, including key-value, document-oriented, column-oriented, and graph databases. 
MongoDB is ranked fifth in popularity among database systems according to the TOPDB database 
index. MongoDB, CouchDB, and Firebase are among the significant document-oriented databases and 
are among the top 14 databases according to the 2021 StackOverflow survey \parencite[p. 41]{Eisermann2022}. 

NoSQL databases differ from traditional relational databases and were developed in the late 2000s.
The term "NoSQL" was first coined by Carlo Strozzi in 1998. There were some pioneers of NoSQL 
databases like Ken Thompson's DBM and Lotus Notes, BerkeleyDB, and GT.M in the early days. 
However, they were not designed for processing large amounts of data.

With the emergence of Web 2.0 and the need to process very large amounts of data, Google became 
a leader in developing technologies like the Map/Reduce framework, the BigTable database, and 
the GFS file system. Other companies like Yahoo, Amazon, MySpace, Facebook, and LinkedIn followed 
suit.

Between 2006 and 2009, many classic NoSQL systems like Cassandra, CouchDB, Neo4j, and MongoDB 
were developed, optimized for processing large amounts of data and horizontally scalable.

In 2009, the term "NoSQL" was used as a buzzword and no longer just stood for "Not Only SQL". 
(Köln, 2020)


% \parencite{Bach2016}              Dokumentenorientierte NoSQL-Datenbanken in skalierbaren Webanwendungen
% \parencite{Eisermann2022}         Dokumentenorientierte Datenbanken
% \parencite{Klettke2016}           Herausforderungen bei der Anwendungsentwicklung mit schema-flexiblen NoSQL-Datenbanken

% \parencite{Faeskorn-Woyke2012}    NoSQL-Datenmodelle und -Systeme
% \parencite{Mongo-Use-cases}
% \parencite{Mongo-Analytics}
% \parencite{Mongo-Deploy-database}
% \parencite{Mongo-E-Commerce}
% \parencite{Mongo-IoT}
% \parencite{Mongo-Mobile}
% \parencite{Mongo-Scale}
% \parencite{Mongo-Was-ist-mongodb}