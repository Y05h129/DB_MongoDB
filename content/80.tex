% !TeX root = ..\main.tex

\section{Position according to CAP}

According to Brewer the CAP theorem includes consistency (C), availability (A), and tolerance to network partitions (P) as its 
components, and it is not possible to achieve all three properties at the same time \parencite{Brewer2000}.

The consistency property can be differentiated into strong consistency, where changes from one node are immediately visible on all nodes,
and eventual consistency, where consistency does not have to be achieved immediately.
Based on that it is necessary to build additional logic to handle potentially inconsistent data \parencite[pp. 4-5]{MongoDB2018}.
\\
MongoDB can provide either strongly consistent or eventually consistent data, depending on the use case \parencite[p. 5]{MongoDB2018}.

Since sharding can be used with MongoDB, data can be stored on multiple nodes and thus be available even if one or more nodes are down or 
not reachable. Thus, when using sharding, MongoDB has high availability even in the event of a partition \parencites{Mongo-Sharding}[p. 7]{Membrey2014}. 

Replica sets lay the foundation for high availability and redundancy. With multiple replications of the data to different database servers a partition tolerance is provided.
Maintaining data copies at different server locations can increase data locality and availability. In addition, data copies can be used for example as backups \parencite{Mongo-Replication}.

MongoDB is considered as a CP database, because it focuses on consistency across the entire system and partition tolerance \parencite{Jayasekara2021}.