% !TeX root = ..\main.tex

\section{Testing MongoDB}

\subsection{Relational example (own example)}
To test MongoDB in practice, an example from the course \enquote{Web-Development II} was used. 
The example was a management system for shopping lists, in which shopping lists are objects with a name and a list of items.
You can check the items on a list, to see if you bought it or not.
In relational databases this example would be implemented using two tables (\enquote{lists} and \enquote{items}) with a one-to-many
relation where each list could contain multiple items.
The test was performed on a locally installed MongoDB server using the native \texttt{mongosh}.

All CRUD operations have been tested.
In the collection \enquote{lists}, we created 2 documents using \texttt{db.lists.insertOne()}. 
Since items are contained in a list, in MongoDB these items are called \enquote{embedded documents} \parencite[p. 6]{Membrey2014}. 
To read all documents in a collection, you use the command: \texttt{db.lists.find()} and for a specific read you add a filter inside the {find()}-Method \parencite{Mongo-read}. 
The Update-Functionality has been tested too, by using \texttt{db.lists.updateOne()}, to update a single document. 
Deleting a document has been tested using \texttt{db.lists.deleteOne()} with a filter as a parameter.


\subsection{User Experience during the test}
Compared to a relational database inserting, querying and deleting documents is very easy, since the documents are basically JSON objects.
At filtering the documents you recognize that the filters are objects (documents) themselves. Additionally, no object-relation mapping was required. 

Since MongoDB implements relations with embedded documents, filtering on a parameter in an embedded document can be harder when compared to a 
relational approach, since the filters have to traverse every level of the containing document \parencite{Mongo-Embedded-Query}. 
This traversing also needs to be done, when overwriting (updating) values in an embedded document. Thus, Read and Update operations can become 
difficult in embedded documents \parencite{Mongo-update}. 