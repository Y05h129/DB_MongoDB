% !TeX root = ..\main.tex

\section{Advantages and Disadvantages}
Like any document oriented database, MongoDB is schemaless and not normalized. Thus, the structure of the stored data can
be highly flexible \parencite{Mongo-Advantages}. On the other hand this leads to more storage space being used \parencite[p. 2]{Cottrell2020}. 

Another advantage is the ability to scale the database horizontally, i.e. by adding more servers instead of upgrading a single server. 
Thus, MongoDB can be run on commodity hardware, which reduces the cost of scaling \parencites[p. 1]{Cottrell2020}{Mongo-Sharding}.
In addition to increased storage capacity and performance, data is stored redundantly in a MongoDB cluster. These aspects are important 
in a big data environment \parencite{Mongo-Advantages}.

The storage of all data in the BSON data format enables easy access from any language, like for example Python, JavaScript, Java, etc. \parencite{Mongo-Advantages}.
Additionally, multi-document ACID transactions are supported to ensure data integrity \parencite{Mongo-ACID}.

Last but not least there are some limits in MongoDB. The maximum size of a document is 16 megabytes
and the maximum levels of nesting is 100 for BSON documents \parencite{Mongo-Limits}.
