% !TeX root = ..\main.tex

\section{API}
To interact with a database, MongoDB provides drivers for a sizeable collection of modern programming languages as well as a shell called \texttt{mongosh}, which itself is a fully
functional Node.JS environment \parencite{Mongo-mongosh, Mongo-API}. The shell and drivers can be used to execute CRUD (create, read, update and delete) operations on documents of the database.
At that, these operations are performed in the context of collections \parencites{Mongo-Crud}[pp. 347-349]{Truica2013}. 

When creating a new document, if the collection of the document does not yet exist, it is implicitly created \parencite{Mongo-insert}. Additionally, each document requires a unique \texttt{\_id} 
field, which is used as the primary key. This field will be generated, if it is not provided when inserting a new document \parencite{Mongo-insert}. Once set, this field cannot be changed and 
is thus immutable \parencite{Mongo-update}. In general, all write operations (creating, updating or deleting documents) in MongoDB are atomic on the level of documents \parencite{Mongo-insert, Mongo-update, Mongo-delete}.

When reading documents from a collection, the MongoDB driver or shell returns a cursor which manages the query results \parencite{Mongo-read}. This cursor can then be iterated to access the 
documents that resulted from the query \parencite{Mongo-cursor}.

% read:
    % cursor 
% concern(s):
    % read isolation / replica sets -> read concern
    % write acknowledgement -> write concern

% if needed name examples for programming languages in line 1 

