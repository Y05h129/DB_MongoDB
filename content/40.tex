% !TeX root = ..\main.tex

\section{Installation}
There is a significant list of packages that you can download on the website of MongoDB: \url{www.mongodb.com}. Depending on operating system and processor of your server, you can choose a version of MongoDB. 
You can decide whether you like a stable version, a released version or a version that is still in development but contains new features \parencite{Membrey2014}.

\subsection{Requirements e.g. configfiles}
There is a reason that there are so many packages for MongoDB. The most important Part is the operating system, just as MacOS, Solaris, Windows, Linux, etc. 
The second part, it depends on, is the bit version of your  system: 32-bit or 64-bit. 
Not to mention, that the software itself, has different versions, depending on the functionality: just as I wrote at the beginning, there are versions which are full versions with some functions and features which has been developed in the past, called a stable version. 
Also there are versions which are released version, which contain the stable version and developed functions (so more functions than the stable version, but has been developed)\dots
And at least the version which is still in development, so anything can be changed with every update you are downloading (not-stable version).

For guarantying that every function and feature can be fully used, there are some supported platforms, with a list including all information, so you know if a version is available on your system\dots
You can view these informations on the MongoDB installation page in the docs: \url{https://www.mongodb.com/docs/manual/installation/}\dots

MongoDB supports a few platforms: 
\begin{itemize}
    \item Amazon Linux V1 and V2
    \item Debian 9, 10, 11
    \item RHEL/CentOS, Oracle Linux, Rocky, Alm a 9.0+, 8.0+, 7.0+, 6.2+
    \item SLES 12, 15
    \item Ubuntu 22.04, 20.04, 18.04, 16.04
    \item Windows Server 2019, 2008/2012/2012R2, 2016
    \item Windows 10
    \item macOS 11, 10.14+, 10.13, 10.12, 
\end{itemize}

\subsection{Linux server}
So if you want to download MongoDB on Linux, you've got two different ways to go:
The first one installs MongoDB through the Repositories, it will install the packages automaticlly. In Ubuntu the default repositories include MongoDB, the version might be out-of-date, so here is the installation-guide with apt-get:
You have to run a few lines in your repository-list in /etc/apt/sources.list \dots
The first one to go is: deB hhtps://downloads-distro.mongodb.org/repo/ubuntu-upstart dist 10gen
Then you need to import MongoDB Inc's public GPG-key, which is used to sign the packages you want to donwload and ensure their consistency. To do that you can use the following apt-key command: 
\dollar sudo 

\subsection{Docker}

\subsection{optional: Windows/MacOS}


\parencite{Mongo-Installation}

\parencite[p. ]{Subramanian2019}