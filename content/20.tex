% !TeX root = ..\main.tex

\section{Comparing document DB and relational DB}





\begin{table}[H]
    \centering
    % \begin{adjustbox}{width=1\textwidth}
        \begin{tabular}{c|c|c}
            \textbf{Category} & \textbf{Relational database}    & \textbf{Document database} \\ \hline \hline
            Language          & SQL                             & NoSQL                      \\ \hline
            Structure of data & structured tables               & unstructured collections   \\ \hline
            Schema            & fixed                           & not fixed                  \\ \hline
            Relations         & through keys                    & through collections        \\ \hline
            Linking data      & with joins                      & without joins              \\ \hline
        \end{tabular}
    %\end{adjustbox}
    \caption{Comparison of relational databases and document databases}
    \label{tab:comparison}
\end{table}
\color{black}

% Relational DB - consistent
% structured data
% data is stored in fixed tables
% tables fixed schema
% each table has unique keys to identify rows
% data in tables has relations with other tables like 1:1 1:n or n:m 
% join queries to achieve data from different

% stable and well known
% optimized storage
% security
% easy to backup

% Documentbased DB - eventually consistent
% unstructured data
% data is stored in documents
% documents does not have to have the same schema
% all data related to one object is stored in one collection
% no joins needed

% flexibility
% high scalability horizontally without more needed ressources
% cost effectiveness

% https://tea-band.com/de/relationale-vs-nicht-relationale-datenbanken/#Dokumentendatenspeicher
% https://datascientest.com/de/sql-vs-nosql-unterschiede-anwendungen-vor-und-nachteile
% https://www.pragimtech.com/blog/mongodb-tutorial/relational-and-non-relational-databases/
% https://www.mongodb.com/de-de/nosql-explained