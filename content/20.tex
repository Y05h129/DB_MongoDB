% !TeX root = ..\main.tex

\section{Comparing document DB and relational DB}

To compare the two types of databases, we will use the following table.

\begin{table}[H]
    \centering
    \begin{adjustbox}{width=1\textwidth}
        \rowcolors{1}{white}{lightgray}
        \begin{tabular}{c|c|c}
            \textbf{Category} & \textbf{Relational database}    & \textbf{Document database} \\ \hline \hline
            Language          & SQL                             & proprietary                \\ 
            group of entries  & table                           & collection                 \\ 
            entry             & row                             & document                   \\ 
            Schema            & fixed                           & not fixed                  \\ 
            Linking data      & join                            & embedded documents         \\ 
        \end{tabular}
    \end{adjustbox}
    \caption{Comparison of relational databases and document databases \parencites[p. 91]{chauhan2019review}{Relational-vs-Document}}
    \label{tab:document-relational-comparison}
\end{table}
\color{black}

The main difference between relational and document databases is the way how data is stored. 
In relational databases, data is stored in tables with fixed schemas. Each table has a unique key to identify the rows \parencites[p. 91]{chauhan2019review}{Relational-vs-Document}. 
And to build relations between tables like 1:1, 1:n or n:m joins are used \parencite[pp. 25-31]{Studer2019}.
In document databases, data is stored in schemaless documents which are grouped in collections.
All data related to one object is stored in one document and no joins are needed to access its data \parencite[pp. 18-20]{Meier2016}.
