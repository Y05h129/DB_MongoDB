% !TeX root = ..\main.tex

\section{Comparing document DB and relational DB}

To compare the two different types of databases, we will use the following table (see \autoref{tab:document-relational-comparison}).

\begin{table}[H]
    \centering
    \begin{adjustbox}{width=1\textwidth}
        \rowcolors{1}{white}{lightgray}
        \begin{tabular}{c|c|c}
            \textbf{Category} & \textbf{Relational database}    & \textbf{Document database} \\ \hline \hline
            Language          & SQL                             & proprietary                \\ 
            group of entries  & table                           & collection                 \\ 
            entry             & row                             & document                   \\ 
            Schema            & fixed                           & not fixed                  \\ 
            Relations         & through keys                    & through collections        \\ 
            Linking data      & join                            & embedded documents         \\ 
        \end{tabular}
    \end{adjustbox}
    \caption{Comparison of relational databases and document databases (\textcite[p. 91]{chauhan2019review}{Relational-vs-Document})}
    \label{tab:document-relational-comparison}
\end{table}
\color{black}

The main difference between relational and document databases is the way how data is stored. 
In relational databases, data is stored in tables with fixed schemas. Each table has a unique key to identify the rows \parencite{chauhan2019review, Relational-vs-Document}. 
And to build relations between tables like 1:1, 1:n or n:m joins are used \parencite[pp. 25-31]{Studer2019}.
In document databases, data is stored in documents. The documents do neither have a fixed nor the same schema. 
All data related to one object is stored in one collection and no joins are needed to access data from different collections \parencite[pp. 18-20]{Meier2016}.


Document databases and relational databases are two common types of database systems widely used 
in software development. Both provide ways to store and retrieve data effectively, but they also 
have their differences. Below are some of the key differences between the two types of databases.

Structure: Relational databases use tables to organize data, while document databases use 
documents that typically exist in the JSON format. Document databases have a more flexible 
structure than relational databases, as they are not bound to a fixed schema structure.

Flexibility: However, the flexibility of document databases comes with a downside: they are less 
suitable for complex queries and evaluating relationships between different data points. In 
relational databases, complex queries and relationships can be mapped using JOIN operations or 
other query methods.

Scalability: Another important factor is scalability. Document databases are often better suited 
for horizontal scaling, as they can easily be split across different servers. Relational databases,
 on the other hand, can be better vertically scaled by upgrading hardware.

Performance: In terms of performance, it can be difficult to make a direct comparison between 
document databases and relational databases. Document databases can often be faster when it comes 
to accessing individual documents, while relational databases are typically faster when it comes 
to complex queries or transactions.

% https://www.mongodb.com/document-databases
% https://www.oracle.com/database/what-is-a-relational-database/


% Relational DB - consistent
% structured data
% data is stored in fixed tables
% tables fixed schema
% each table has unique keys to identify rows
% data in tables has relations with other tables like 1:1 1:n or n:m 
% join queries to achieve data from different

% stable and well known
% optimized storage
% security
% easy to backup

% Document based DB - eventually consistent
% unstructured data
% data is stored in documents
% documents does not have to have the same schema
% all data related to one object is stored in one collection
% no joins needed

% flexibility
% high scalability horizontally without more needed resources
% cost effectiveness

% https://tea-band.com/de/relationale-vs-nicht-relationale-datenbanken/#Dokumentendatenspeicher
% https://datascientest.com/de/sql-vs-nosql-unterschiede-anwendungen-vor-und-nachteile
% https://www.pragimtech.com/blog/mongodb-tutorial/relational-and-non-relational-databases/
% https://www.mongodb.com/de-de/nosql-explained