% !TeX root = ..\main.tex

\section{Comparing document DB and relational DB}

To compare the two different types of databases, we will use the following table (see \autoref{tab:document-relational-comparison}).

\begin{table}[H]
    \centering
    \begin{adjustbox}{width=1\textwidth}
        \rowcolors{1}{white}{lightgray}
        \begin{tabular}{c|c|c}
            \textbf{Category} & \textbf{Relational database}    & \textbf{Document database} \\ \hline \hline
            Language          & SQL                             & proprietary                \\ 
            group of entries  & table                           & collection                 \\ 
            entry             & row                             & document                   \\ 
            Schema            & fixed                           & not fixed                  \\ 
            Relations         & through keys                    & through collections        \\ 
            Linking data      & join                            & embedded documents         \\ 
        \end{tabular}
    \end{adjustbox}
    \caption{Comparison of relational databases and document databases (\textcite[p. 91]{chauhan2019review}{Relational-vs-Document})}
    \label{tab:document-relational-comparison}
\end{table}
\color{black}

The main difference between relational and document databases is the way how data is stored. 
In relational databases, data is stored in tables with fixed schemas. Each table has a unique key to identify the rows \parencite{chauhan2019review, Relational-vs-Document}. 
And to build relations between tables like 1:1, 1:n or n:m joins are used \parencite[pp. 25-31]{Studer2019}.
In document databases, data is stored in documents. The documents do neither have a fixed nor the same schema. 
All data related to one object is stored in one collection and no joins are needed to access data from different collections \parencite[pp. 18-20]{Meier2016}.


% Relational DB - consistent
% structured data
% data is stored in fixed tables
% tables fixed schema
% each table has unique keys to identify rows
% data in tables has relations with other tables like 1:1 1:n or n:m 
% join queries to achieve data from different

% stable and well known
% optimized storage
% security
% easy to backup

% Document based DB - eventually consistent
% unstructured data
% data is stored in documents
% documents does not have to have the same schema
% all data related to one object is stored in one collection
% no joins needed

% flexibility
% high scalability horizontally without more needed resources
% cost effectiveness

% https://tea-band.com/de/relationale-vs-nicht-relationale-datenbanken/#Dokumentendatenspeicher
% https://datascientest.com/de/sql-vs-nosql-unterschiede-anwendungen-vor-und-nachteile
% https://www.pragimtech.com/blog/mongodb-tutorial/relational-and-non-relational-databases/
% https://www.mongodb.com/de-de/nosql-explained